\documentclass{article}

\usepackage[margin=1in]{geometry}
\usepackage{titlesec}
\usepackage[hidelinks]{hyperref}
\usepackage{color}
\usepackage{fontawesome}
\usepackage{multicol}
\usepackage{enumitem}
\usepackage{booktabs}
\usepackage{array}
\usepackage{makecell}
\usepackage{parskip}

\newcommand{\n}[0]{\\[\baselineskip]}
\newcolumntype{R}{>{\raggedleft\arraybackslash}p{4cm}}

\setlist[itemize]{topsep=0pt}

\titleformat{\section}{\normalfont\bfseries}{\thesection}{0pt}{}[{\titlerule[0.8pt]}]


%--------------------BEGIN DOCUMENT----------------------
\begin{document}

\pagestyle{empty} % non-numbered pages


%--------------------TITLE-------------
\par{\centering
		{\Large \textbf{Sizhe Yuen}
	}\par}

\begin{center}
3 Albert Road, Chelsfield, Orpington, BR6 6JG \\
\href{https://github.com/12yuens2}{\faGithub\ 12yuens2}
\ \ \ \href{https://www.linkedin.com/in/sizhe/}{\faLinkedin\ sizhe/}
\ \ \ \href{mailto:sizhe1007@gmail.com}{\faEnvelopeO\ sizhe1007@gmail.com}
\ \ \ \faMobile\ (+44) 07481 116190
\end{center}


%--------------------EDUCATION-----------------------------------
\section*{Education}
\begin{tabular}{r|p{13.5cm}}
\textsc{2019-Present} & \textbf{University of Southampton}, PhD in Engineering
\\
& \\
& \textit{Epigenetics crossover for multi-objective optimisation}

    % Analysed existing Evolutionary Computation algorithms
    % based on evolutionary concepts from the Extended Evolutionary Synthesis, and 
    % identified a gap in epigenetic inheritance.
    % A novel epigenetic blocking mechanism was then developed and a detailed benchmarking 
    % suite created to test and compare existing algorithms with the new
    % mechanism. Hyperparameter optimisation was further studied
    % find the best performing results on dynamic multi-objective optimisation
    % problems. The final results were applied to a real world voyage optimisation
    % problem to reduce fuel consumption and voyage times.

    \begin{itemize}
        \item Analysed existing Evolutionary Computation algorithms
    based on evolutionary concepts from the Extended Evolutionary Synthesis, and 
    identified a gap in epigenetic inheritance.
    \item Developed a novel epigenetic blocking mechanism and created a detailed benchmarking 
    suite to test and compare existing algorithms with the new mechanism.
    \item Investigated the use of hyperparameter optimisation to find the best 
    performing results on dynamic multi-objective optimisation problems.
    \item Applied the final results to a real world voyage optimisation
    problem to reduce fuel consumption and voyage times of international shipping journeys.
    \end{itemize}
    % to determine their differences and suitability to a new epigenetic mechanism.
    % A novel epigenetic blocking mechanism based on gene regulation is
    % developed and tested on the identified algorithms. Performance on static
    % and dynamic multi-objective problems show the improvement the epigenetic mechanism
    % can make in dynamic optimisation, with improved performance across the 
    % duration of the optimisation process. Further study and comparison of the
    % hyperparameters indicate the mechanism is both problem and algorithm specific,
    % requiring different hyperparameters depending on the properties of a problem
    % to achieve the best performance.
\\
\textsc{2014-2019} & \textbf{University of St. Andrews}, MSci (Hons) Computer Science First Class
\\
\textsc{2012-2014} & \textbf{Shatin College, Hong Kong}, Bilingual IB Diploma
\end{tabular}


%-------------PUBLICATIONS-------------------
\section*{Publications and presentations}
S. Yuen, T. H. G. Ezard, A. J. Sobey.
\textbf{Epigenetic Opportunities for Evolutionary Computation.} 
In: \textit{The Journal of Royal Society Open Science} 10.5 (2023) 
https://doi.org/10.1098/rsos.221256
\\

S. Yuen, T. H. G. Ezard, A. J. Sobey.
\textbf{The effect of epigenetic blocking on dynamic multi-objective optimisation problems.} 
In: \textit{Proceedings of the Genetic and Evolutionary Computation Conference Companion 
(GECCO ’22).} (2022)
https://doi.org/10.1145/3520304.3529022


Presented at the MIT Quest for Intelligence seminar, July 2022.
\\

S. Yuen, T. H. G. Ezard, A. J. Sobey. \textbf{Comparing the performance of
  genetic algorithms and particle swarm optimisation algorithms for
  multi-objective optimisation problems.}
  (In preparation for \textit{Evolutionary Computation})



%--------------------PROJECTS-----------------------------------
\section*{Previous projects}
% \textbf{Settlers of Catan} \textit{Junior Honours project} \\
% Year long team project to create a digital Settlers of Catan board game.
% Worked in a team of 4 where I was responsible for the backend and AI in Java.
% Participated in inter-team discussions for a communication protocol that allowed a game
% to be played over different implementations.

%\textbf{Unix file system} & Implemented a simple \texttt{POSIX} file system in C using unqlite key-value store for persistent storage. Features included a tree-structured directory hierarchy with sub-directories and file, directory and meta-data creation, update and deletion.\n

%\textbf{Multicast file distribution protocol} & Designed and implemented a protocol to reliably and efficiently replicate whole files on persistent storage across multiple nodes in C. Made design decisions regarding the use of UDP multicast and TCP control messages based on observations during initial deployment.\\

%\textbf{Reversi/Othello} & Implemented the Othello board game with Haskell in a team of 3.\\

\textbf{Dota 2 player prediction} \textit{MSci dissertation} \\
Used machine learning algorithms to predict the player behind the keyboard through 
behavioural features such as mouse movements and in-game decisions. Features were 
extracted and parsed from match replays, then clean and refined through a data pipeline.
Finally, the performance of different combinations of features and machine learning 
techniques were compared.
\\

\textbf{Graph matching with lobsters} \textit{BSc dissertation} \\
Applied computer vision algorithms from opencv on images of lobsters to represent their bodies as
attributed graphs by detecting points of interest. Graph matching techniques were then
used to discover and measure properties such as the lobster's size and maturity.

%--------------------WORK EXPERIENCE-----------------------------------
\section*{Work experience}
%\renewcommand{\arraystretch}{1.5}
\textbf{University of Southampton} \hfill \textsc{January 2020-June 2022}
\begin{itemize}
    \item \textit{Software Engineering Group Supervisor} - Led groups of
    second year undergraduate students on the software engineering group module lasting
    three months at a time. Reviewed and provided feedback on multiple iterations
    of the software engineering process with agile development methods, acting as
    both a critical friend and project customer for the students.
    \item \textit{MSc Mentor} - Mentored groups of masters students,
    providing support, guidance, and presenting training material once a week. Training material
    included topics of reading, writing, citing, presenting, and time management.
    \item \textit{Lab Demonstrator} - Demonstrated on undergraduate modules
    for advanced techniques with Java, and functional programming with Haskell.
\end{itemize}
\ \\
\textbf{Skyscanner, Edinburgh} \textit{Software Engineering Intern}
\hfill
\textsc{June 2018-September 2019} \\
Part of the data platform team which provided logging, monitoring and alerting services for
other teams in the company, centred around real time data. The stack consisted of open source
technologies such as Kafka, OpenTSDB, Bosun, Grafana, and ELK (Elasticsearch, Kibana)
deployed via containers on AWS.
%through a variety of methods including Amazon Machine Images and Ansible.
\newline

\textbf{New Modern Technology Ltd. Hong Kong} \textit{Programmer Trainee (Summer)}
\hfill
\textsc{June-July 2017} \\
Worked on development for a financial analysis system including user interface,
backend functionality and data maintenance with frameworks such as J2EE, Hibernate and GWT.
Supported a production software management suite with compatibility testing and maintenance.
\newline

\textbf{KPMG China, Hong Kong} \textit{Trainee}
\hfill
\textsc{May-August 2016} \\
Wrote VBA modules in Word to create custom toolbars with useful macros.
Designed HTML pages with CSS and JS for new branding and was responsible for
internet and intranet updates for the multimedia team.
\newline

\textbf{HongKong International Terminals} \textit{Summer Intern}
\hfill
\textsc{June-August 2015} \\
Wrote VBA macros in Excel to transform shipping data logs, helped business analyst
with writing user specifications for system updates, and worked on a project with other
interns to raise fitness awareness in the workplace.


%--------------------SKILLS-----------------------------------
\section*{Languages and Frameworks}
\begin{tabular}{r|p{15cm}}
\textsc{Programming Languages} & Java, Python, C, Haskell, Javascript, Go\\

\textsc{Tools and Frameworks} & Linux, Git, pandas, scikit-learn, \LaTeX, Travis-CI, Slurm %, Bosun, AWS
\end{tabular}

%--------------------AWARDS-----------------------------------
% \section*{Awards}
% \setlength\multicolsep{0pt}
% \begin{multicols}{2}
% \begin{itemize}
% \item University of St Andrews Dean's List (2019, 2018, 2017, 2016, 2015)
% \item Winner of the Morgan Stanley Challenge at Hack the Burgh, Edinburgh (2017)
% \end{itemize}
% \columnbreak
% \begin{itemize}
% \item Medal for performance in Programming Projects module (2015)
% \item Shatin College Computer Science Outstanding Student Award (2014)
% \end{itemize}
% \end{multicols}



%--------------------ACTIVITIES-----------------------------------
% \section*{Other activities}
% \begin{tabular}{R|p{15cm}}
% \textbf{Hackathons and competitions} & Participated in multiple hackathons and competitions such as Google Hashcode, UKIEPC, StacsHack and Hack the Burgh. \n
% 
% \textbf{Community service} & Worked as a volunteer at Fung Yuen Butterfly Reserve from 2010 to 2014. \n
% 
% \textbf{Chinese calligraphy} & Learned and practice Chinese calligraphy since 2005. Write Fai Chun annually during Chinese New Year for friends and family. 
% \end{tabular}

\end{document}
